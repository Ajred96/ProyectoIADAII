\subsection{Implementación}
    Respecto a la implemenctación de \textbf{modciPD.py} es importante resaltar que esta utiliza una estrategia bottom-up dado a que la solución se genera  a partir de las soluciones de los subproblemas más pequeños hasta el problema más grande, a su vez esta sigue los siguientes pasos:
    \begin{enumerate}
     
    \item \textbf{Inicialización de DP:}
    \begin{itemize}
        \item Se crea la tabla $dp$ como una lista de diccionarios, la cual será la encargada de guardar los pasos que lleven a la solución óptima.
        
    \end{itemize}
    
    \item \textbf{Actualización de estados:}
    \begin{itemize}
        \item Para cada grupo dentro de la red, se recorren las posibles decisiones (pasos) y estas son almacenadas solo si cumplen con el criterio de que al aplicar las mismas no se puede superar el esfuerzo máximo proporcionado por la red. 
    \end{itemize}
    
    \item \textbf{Selección de la estrategia óptima:}
     \begin{itemize}
        \item Una vez procesados todos los grupos, se analiza el último nivel de la tabla $dp$ (el cual corresponde a $dp[n]$, allí se encuentran todas las estrategias posibles que llegaron hasta el final (significa que tomaron una decisión válida para todos los grupos respetando el esfuerzo máximo $(R_{max})$).
        \item Cada una de estas estrategias esta asociada a un esfuerzo utilizado y un número total de personas que no fueron moderadas.
        \item La función \textbf{seleccionar\_mejor\_estrategia} revisa todas las estrategias anteriormente mencionadas y para cada una de ellas recupera el valor acumulado del conflicto y las decisiones tomadas en cada grupo.
        \item Luego, se calcula un conflicto promedio de cada estrategia (a partir de la división del conflicto acumulado entre el número de grupos) y se selecciona la estrategia que tenga el conflicto más bajo.
        \item En esta etapa se determina cuál fue la mejor estrategia y se reconstruyen las decisiones que llevaron hasta ese punto.
    \end{itemize}    
\end{enumerate}
Finalmente, se utiliza la función \textbf{calcular\_ci\_esfuerzo} para obtener el valor real del conflicto interno de la estrategia seleccionada y se retorna una tupla compuesta por: el conflicto interno total alcanzado por la estrategia, el esfuerzo total requerido para aplicarla y la lista de decisiones tomadas para moderar a los agentes.