\textbf{Conclusión:}
\begin{itemize}
    \item Si se prioriza el tiempo de ejecución, el algoritmo voraz es ideal, resolviendo virtualmente al instante.
    \item Para situaciones donde la calidad de la solución es crítica, puede preferirse programación dinámica, aunque con tiempos más elevados.
    \item Fuerza bruta solo es viable para instancias pequeñas, siendo impráctico para escenarios reales por su excesivo tiempo.
\end{itemize}

La siguiente tabla presenta una comparación general entre los tres enfoques implementados: fuerza bruta, programación dinámica y algoritmo voraz. Se analizan aspectos clave como eficiencia, escalabilidad y adecuación al problema:

\begin{table}[H]
    \centering
    \renewcommand{\arraystretch}{1.5}
    \begin{tabular}{C{4cm} C{3cm} C{3.5cm} C{3.5cm}}
        \toprule
        \textbf{Criterio}                              & \textbf{Fuerza Bruta}                   & \textbf{Programación Dinámica} & \textbf{Algoritmo Voraz}              \\
        \midrule
        Complejidad Temporal                           & \( O(n(k+1)^n) \)                       & \(O(n \cdot R_{\text{real}} \cdot k_{\text{real}})\)                 & \( O(n \log n) \)                     \\
        Complejidad Espacial                           & \( O((k+1)^n) \)                        & \( O(n^2 \times R_{max} \times k)\)                 & \( O(n) \)                            \\
        ¿Siempre da la solución óptima?                & Sí                                      & Sí                             & No                                    \\
        ¿Es capaz de manejar gran cantidad de agentes? & No                                      & Moderada                 & Alta                                  \\
        Aplicación recomendada                         & Problemas pequeños                      & Problemas moderados                      & Problemas grandes                     \\
        Eficiencia en Tiempo                           & Muy baja (crece exponencialmente)       & Moderada (tiempo polinomial)                      & Alta (tiempo lineal o logarítmico)    \\
        Eficiencia en Espacio                          & Muy baja, requiere mucho almacenamiento & Moderada, requiere una cantidad de almacenamiento moderada                      & Alta, requiere poco espacio adicional \\
        \bottomrule
    \end{tabular}
    \caption{Comparación general entre algoritmos.}
\end{table}